%
%	Many revisions ago, this CV template came from: 
%
% Jason R. Blevins <jrblevin@sdf.lonestar.org>
% http://jrblevin.freeshell.org
% Durham, December 12, 2006
%
%%---------------------------------------------------------------------------%
\documentclass[margin,line,letterpaper]{res}

\usepackage[colorlinks=true, pdfstartview=FitV, linkcolor=blue, citecolor=blue, urlcolor=blue]{hyperref}

%%===========================================================================%%

\begin{document}

%---------------------------------------------------------------------------
% Document Specific Customizations

% Make lists without bullets and with no indentation
\setlength{\leftmargini}{0em}
\renewcommand{\labelitemi}{}

% Use large bold font for printed name at top of pages
\renewcommand{\namefont}{\large\textbf}

%---------------------------------------------------------------------------

\name{Nathan Thomas Moore}

\begin{resume}

\begin{ncolumn}{2}
  Department of Physics		&   nmoore@winona.edu \\
  Winona State University	&  \url{http://course1.winona.edu/nmoore/}		\\
  Winona, MN 55987			&  \url{https://github.com/ntmoore}			\\
  Phone: (507) 457-5611		&  \url{http://orcid.org/0000-0001-8590-8349}	\\
\end{ncolumn}


%---------------------------------------------------------------------------

\section{\bf Education}
\begin{itemize}
\item Ph.D. Physics, University of Minnesota 2006.\\
	 PhD Thesis: ``Knot Entropy,'' thesis advisor A.Y. Grosberg.
\item B.S. Applied Physics, Grove City College 2000
\end{itemize}
%---------------------------------------------------------------------------

\section{\bf Timeline}

\begin{table}[h]
\begin{center}
\begin{tabular}{lr}
Professor of Physics&2015-present\\
Physics Chairperson &2012-13, 2019-2022\\
Associate Professor of Physics&2010-2015\\
Assistant Professor of Physics&2005-2010\\
Winona State University&Winona, Minnesota\\
\\
Blue Gene Science Application Analysis &  2005\\
IBM & Rochester, Minnesota\\
\\
Research and Teaching Assistant & 2000-2005\\
Physics Department  and Army HPCRC & University of Minnesota, Minneapolis\\
\end{tabular}
\end{center}
\end{table}

%\section{\bf Teaching}
%
%\begin{table}[h]
%\begin{center}
%\begin{tabular}{ll}
%Physics 115 & Conceptual Physics\\
%Physics 141 & Physics for Future Presidents\\
%Physics 150 & Engineering for the Modern World\\
%Physics 180 & Investigative Science (aka Science Education 201)\\ %(Spring 2008, 2009)
%Physics 201-2 & Algebra-based Introductory Physics with labs\\ %(Fall 2005 to Fall 2006, Summers 2007-2011)
%%Physics 201 & ``First-year cohort'' section (Fall 2009, 2010)\\
%Physics 221-2 & Calculus-based Introductory Physics with labs\\ %(Spring 2006 to Fall 2009, Spring 2011)
%Physics 320 & Computational Physics\\ %(Fall 2006, 2008, Spring 2010)
%Physics 321 & Computerized Data Acquisition and Analysis\\ %(Fall 2010)
%Physics 330 & Electronics with labs\\
%Physics 332 & Computer Organization (Digital Circuits) with labs\\ %(Fall 2006 to Fall 2010)
%Physics 333 & Microprocessor Electronics with labs\\
%Physics 340 & Modern Physics\\
%Physics 345 & Thermodynamics and Statistical Mechanics\\ %(Spring 2006, 2008, 2010)
%Physics 350 & Classical Mechanics\\ %(Fall 2005, Fall 2007)
%Physics 370 & Optics\\
%Physics 451 & Quantum Mechanics \\%(Spring 2007, 2009)
%%Chemistry 108 & variant funded by MNSCU CTL grant, Fall 2008\\
%%EFRT 099 & Elementary Education Field Experience, Fall 2011 \\
%%CME 370 & Heat and Mass Transfer, Spring 2013\\
%\end{tabular}
%\end{center}
%\end{table}
%
%%===========================================================================%%

\section{\bf Publications}

``How many acres of potatoes does a society need?,''
Nathan T. Moore.
Submitted.
\url{https://arxiv.org/abs/2301.06637v1}

``Don't throw that video away! Reference Frames can fix Video Analysis with a Moving Camera,''
Nathan T. Moore.
IOP Physics Education. 
59 (2024) 015029.\\
\url{https://arxiv.org/abs/2301.00013}


``Inexpensive Student-fabricated Solar Panels and Some Related Classroom Measurements,"
Nathan T. Moore and Carl D. Ferkinhoff.
Submitted.\\
\url{https://arxiv.org/abs/1712.04029}

``A model for including Arduino microcontroller programming in the introductory physics lab,''
Andrew J. Haugen and Nathan T. Moore. \\
Submitted.
\url{http://arxiv.org/abs/1407.7613}

``Small Oscillations via Conservation of Energy,''
Tia Troy, Megan Reiner, Andrew J. Haugen, and Nathan T. Moore. \\
(IOP) Physics Education, vol. 52, no. 6, 2017.
\url{http://arxiv.org/abs/1407.5243}

``Using Cognitive Acceleration Materials to Develop Pre-Service Teachers' Reasoning and Pedagogical Expertise," 
Nathan Moore, Jacqueline O'Donnell, and Dennis Poirier.
2012 ASQ Advancing the STEM Agenda in Education, the Workplace and Society.
(peer reviewed)
\url{http://asq.org/qic/display-item/index.html?item=34852}

``Computational Physics and Reality: Looking for Some Overlap at the Blacksmith Shop",
Nathan Moore and Nicole Schoolmeesters,
submitted.\\
\url{http://arxiv.org/abs/0904.3960}

``Using Garlic As A Far-Transfer Problem of Proportional And Probabilistic Reasoning'',
Nathan Moore and John Deming, Mathematics Teacher, August 2010.\\
\url{http://arxiv.org/abs/0811.2133}

``Measuring the 2D Vector Aspect of Momentum Using Only One Dimension'',
Andrew Ferstl and Nathan Moore,
submitted.\\
\url{http://arxiv.org/abs/0803.4142/}

``Abundance of unknots in various models of polymer loops'',
N.T. Moore and A.Y. Grosberg,
J. Phys. A: Math. Gen. 39, 9081, (2006).\\
\url{http://arxiv.org/abs/cond-mat/0604225/}

``On the Limits of Analogy Between Self-Avoidance and Topology-Driven Swelling of Polymer Loops'',
N.T Moore and A.Y. Grosberg,
Phys. Rev. E 72, 061803 (2005).\\
\url{http://arxiv.org/abs/cond-mat/0506786}

``Topologically Driven Swelling of a Polymer Loop'',
N.T. Moore, R.C. Lua, A.Y. Grosberg.
Proc. Natl. Acad. Sci. USA 101(37), 13431-13435, (2004).\\
\url{http://arxiv.org/abs/cond-mat/0403419/}

``Under-knotted and over-knotted polymers: 1. Unrestricted loops'',
N.T. Moore, R.C. Lua, A.Y. Grosberg,
in Physical and Numerical Models in Knot Theory, Including Applications to the Life Sciences,
        Series on  Knots and Everything 36 363-384 (World Scientific)\\
\url{http://arxiv.org/abs/cond-mat/0403457/}

``Under-knotted and over-knotted polymers: 2. Compact self-avoiding loops'',
R.C. Lua, N.T. Moore, A.Y. Grosberg,
in Physical and Numerical Models in Knot Theory, Including Applications to the Life Sciences,
        Series on  Knots and Everything 36 385-398 (World Scientific)\\
\url{http://arxiv.org/abs/cond-mat/0403413/}

%%===========================================================================%%

%\section{\bf Talks and Commentary}

%\begin{itemize}
%\item \textbf{``Fun with incomplete data''}, University of Minnesota College in the Schools instructors, August 2019.

%\item \textbf{``Chronic Wasting Disease: Please shoot all the deer so we can eat Cheerios in peace!''} Winona Nerd Nite, June 2018, Winona, MN.

%\item \textbf{SEPCI Workshop}, organized by Lee Schmitt, Hamline University.  Forces and Bridges workshop leader. June 2016, Anoka, MN.

%\item \textbf{``Cognitive Acceleration in the Physics Classroom,''} a talk for College in the Schools teachers at the University of Minnesota.  27 February 2015 at the East Bank of UMN.  

%\item 
%\textbf{``Is School-Lunch More Effective than School?}
%The relevance of Proportional Reasoning in Physics (and other science classes)'' Physics Strand Speaker at MnSTA's MnCOSE15 meeting.  20 February 2015, Mankato, MN.  Also given at a Winona Nerd Nite, May 2019.
%
%\item
%\textbf{``Tree Rings, Thyroid Disorders, and the Distribution of Problem-Solving Ability in Children.''}
%16 February 2015, Rochester Century High School, Rochester MN.
%
%\textbf{``Adapting Modeling Instruction to DIY Arduino (microcontroller) Lab Equipment Development,"} and \textbf{``Student-built lab equipment via an Arduino and Modeling Instruction,"} and \textbf{``Arduino in an Undergraduate Lab Curriculum and Applications
%"}, with Tia Troy, Megan Reiner, and Andrew J. Haugen.  
%National AAPT meeting, Minneapolis, July 2014.
%
%\item
%\textbf{``Modeling, Making, DIY, and Arduino,"}
%Minnesota Science Teachers' Association MnCOSE14 meeting, 21 February, 2014.  
%
%\item
%\textbf{```When Will We Make Potions?' A Description of Elementary Science Clubs in Winona,"}
%WSU's Consortium for Liberal Arts and Science Promotion (CLASP), 30 October, 2013.
%
%\item \textbf{``Strategies for Reasoning Ability Growth in Pre-service Teacher Preparation: Cognitive Acceleration, Science Clubs, and STEM Elementary Schools''}
%Thomas Edison High School, 7 May, 2013.
%
%\item \textbf{``Designing Curriculums for STEM Elementary Schools"}, University of Minnesota Physics Education Seminar, 8 March 2013.
%
%\textit{Abstract: Suppose you accept a request to help design a ``STEM" elementary school. What would you include in the curriculum? What outcomes would you value? Is it possible to make such a school better than the average institution through specific curricular choices? In the talk, I will talk about my experience in such a situation and the way it has altered my perception/practice of education. Specific topics touched on include measures of (Piagetian) reasoning ability, a detailed and useful taxonomy of reasoning ability, Shayer and Adey's program of ``Cognitive Acceleration through Science Education," and after-school elementary science clubs (Zvonkin's math circles).}
%
%\item \textbf{``Strategies for Reasoning Ability Growth in Pre-service Teacher Preparation: Cognitive Acceleration, Science Clubs, and STEM Elementary Schools''}
%Minnesota Science Teachers' Association Meeting, 22 February, 2013. 
% 
%\item \textbf{``Exploring our Common Questions: Inquiry about Teaching and Learning at MN's Public Colleges and Universities",}
%Panelist, University of Minnesota Postsecondary Teaching and Learning Research Series,
%8 November, 2012.
%
%
%\item \textbf{``Using Cognitive Acceleration Materials to Develop 
%Pre-service Teachers' Reasoning and Pedagogical Expertise"}, 
%Education Division of the American Society for Quality (ASQ) meeting, Summer 2012, University of Wisconsin-Stout.
%
%\textit{\textbf{Abstract:}
%The talk outlines two approaches taken at Winona State University (WSU) to increase the reasoning ability of pre-service elementary education majors through exposure to the Cognitive Acceleration materials produced by Shayer, Adey, and collaborators in the UK.}
%
%
%\item \textbf{Global Physics Department, Cognitive Acceleration}  Summary: last night we spoke with Nathan about what he and collaborators are doing to study the effects of increasing students cognitive ability. We talked about the Lawson test, UK education, middle schools, and lots of other ideas. 5 October 2011.
%
%\item 
%\textbf{The Arduino, a cheap micro-controller with multiple uses in the physics curriculum}
%, Joint Wisconsin and Minnesota American Association of Physics Teachers Meeting, University of Wisconsin at River Falls, 30 October 2010.
%
%%\textit{\textbf{Abstract:}
%%The talk will introduce the Arduino microprocessor (and programming) system, an exciting piece of equipment which is popular in hobbyist electronics (aka ``Maker") circles.  The system is quite cheap, easy to program, and has a wealth of open-source applications already developed and available (from accelerometers to GPS to breathalyzers...).  A few upper and lower division activities using the Arduino will be demonstrated.
%%}
%
%%\item \url{http://www.winonapost.com/stock/functions/VDG_Pub/searchdetail.php?choice=37456}{\textbf{``Of Math and Mistakes"}}, Letter to the Editor, Winona Post, 1 August, 2010. 
%
%\item %\url{http://toulouse.physics.winona.edu/dokuwiki/doku.php?id=talks:clasp_f09}{
%\textbf{``The Physics of Cold-Weather Gardening"}%}
%, WSU's Consortium for Liberal Arts and Science Promotion (CLASP), 2 December, 2009.
%
%%\textit{\textbf{Abstract:} Minnesota is a nice place to live, but as we all know, in the winter the temperature gets lower than a typical backyard garden can survive. Over the last few years I've played around with the techniques for season-extension and winder vegetable harvest described in Eliot Coleman's \url{http://books.google.com/books?id=QMHdDgkRjDkC}{``Four-Season Harvest''}. On the whole, I've found Coleman's ideas to be quite sound, and I plan to talk about how to grow and harvest green vegetables in December, in Minnesota.}
%
%\item %\url{http://toulouse.physics.winona.edu/dokuwiki/lib/exe/fetch.php?media=talks:apsps_fall2009.pdf}{
%\textbf{``Underprepared is underserved: a first-year math and science cohort class"}%}
%, Inaugural meeting of the Prairie Section of the American Physical Society. 
%The University of Iowa, 12-14 November 2009. Session $Q2.00004$.  
%
%%\textit{\textbf{Abstract:} 
%%A troubling fraction of students in STEM majors flounder at the introductory level. The most compelling reason for this is a lack of adequate intellectual preparation. On a fundamental level, science is done by thinking critically about the natural world.  Students with weak quantitative reasoning skills will struggle in quantitative science fields. The talk will discuss the depth of the problem, a teaching strategy  implemented at Winona State University which is designed to enhance these skills, and initial results, indicating a substantial increase in student ability and retention in STEM fields. }
%
%
%\item 
%\textbf{``A reminder that meat comes wrapped in fur and feathers''},
%Guest Opinion, Winona Daily News, 29 August, 2009.
%
%
%\item 
%\textbf{``Blacksmithing and your GeForce Video Card''}%}
%, WSU Physics Department Occasional Seminar, September 2009.
%
%%\textit{\textbf{Abstract:} In the talk I'll describe a Physics 222 activity, a Blacksmithing workshop at DreamAcres Farm, which turned into a paper that Nicole Schoolmeesters (WSU '09) and I recently had accepted for publication. After introducing the basic numerical model used in the paper, I'll talk about my hopes to port the code to run on a commodity graphics card via the NVIDIA CUDA library.  }
%%%\url{http://toulouse.physics.winona.edu/dokuwiki/lib/exe/fetch.php?media=occasional_seminar:wsu_physoccsem_2009_09_03.m4a}{podcast}}
%
%\item %\url{http://toulouse.physics.winona.edu/dokuwiki/doku.php?id=talks:blacksmithing_mnaapt_s09}{
%\textbf{``Blacksmithing, introductory physics, and computer programming''}
%%}
%, Minnesota American Association of Physics Teachers meeting at the University of St. Thomas, April 25, 2009.
%%
%%\textit{\textbf{Abstract:}
%%In the talk, I will describe a computational model of heat flow within a metal bar created by students in my calculus-based introductory physics course. After creating the simulation, students took a blacksmithing seminar and had a chance to work with iron and take data on the heating of iron in a coke forge. On their return to campus, students revised their computational models in light of their experimental data. 
%%}
%%
%
%\item 
%%\url{http://toulouse.physics.winona.edu/dokuwiki/doku.php?id=talks:learning_games_iteach_2009}{
%\textbf{``A Different Sort of Homework: Student-Created Simulations using VPython''}
%%}
%, MNSCU iTeach Conference, Minneapolis Community and Technical College, February 27, 2009.
%%
%%\textit{\textbf{Abstract:}
%%Lately, the presenter's introductory physics students have tackled programming problems as homework. Solutions to these problems are student-created, interactive 3-D simulations, written in VPython, an easy-to-learn programming language. The presenter will describe the results, in terms of student learning, that they have seen from this curricular addition. Installation of VPython will be covered (so bring your laptop!), as well as the creation of a simulation.
%%}
%
%
%\item \textbf{``Underprepared is Underserved: A Freshman Math and Science Cohort Class''}, 
%MNSCU iTeach Conference, Minneapolis Community and Technical College, February 27, 2009.
%Given with John Deming, WSU Chemistry.  We gave a similar talk at a meeting of the 2--year College Chemistry Consortium at Rochester Community and Technical College, September 25, 2009.
%%
%%\textit{\textbf{Abstract:}
%%A troubling fraction of students in STEM majors flounder at the introductory level. The most compelling reason is a lack of adequate intellectual preparation. On a fundamental level, science is done by thinking critically about the natural world. Students with weak quantitative reasoning skills will struggle in quantitative science fields. Presenters will discuss the depth of the problem, a teaching strategy designed to enhance these skills, and their efforts to help these students become competitive.
%%}
%
%
%\item 
%%\url{http://toulouse.physics.winona.edu/dokuwiki/lib/exe/fetch.phpmedia=talks:maapt_fall08_bread_oven_heat_flow.pdf}{
%\textbf{``A Molecular Cartoon of Heat Transfer''}
%%}
%, Minnesota American Association of Physics Teachers meeting at Gustavus Adolphus College, October 25, 2008.
%%
%%\textit{\textbf{Abstract:}
%%The talk describes a thermodynamics problem used in a University Physics 2 class.  The problem solved by students was to model the flow of heat through a block of material by simulating the ``motion" of energy quanta through a solid.  Students tested the qualitative validity of their simulations by comparing energy density profiles from their simulations to an experimental data set taken in the departmental bread oven (which was built by a previous Thermodynamics class).
%%}
%
%\item 
%%\url{http://toulouse.physics.winona.edu/dokuwiki/doku.php?id=faculty_devel_talk_2008:start}{
%\textbf{``Better than Sliced Bread: Using a wiki to reduce intellectual clutter"}
%%}
%, Faculty Development Workshop, 21 August, 2008.
%%
%%\textit{\textbf{Abstract:} 
%%Nathan Moore has found a wiki 
%%%(at \url{http://toulouse.physics.winona.edu}{http://toulouse.physics.winona.edu}) 
%%to be a  nearly indispensable tool that has made the organization of his academic life much easier.  In the  first part of this session, Nathan will tell a few stories about what he think a wiki is "good for".  In  the second half of the session, Ken Graetz, the E-Learning Director, and Nathan will host a hands-on "howto" workshop that will allow participants to set up wiki's for their teaching or research needs. }
%
%\item 
%%\url{http://minnesota.publicradio.org/display/web/2008/06/04/winonacanoe/}{
%\textbf{``Without bridge, Winona commuter trades car for canoe''}, All Things Considered, Minnesota Public Radio, 4 June, 2008.
%%
%%\textit{\textbf{Abstract:}  Jeremy Smith says his commute time quadrupled now that Minnesota has closed the Highway 43 bridge over the Mississippi River at Winona. Smith lives in Wisconsin, but works at a canoe shop in Winona.  On Thursday he plans to take a canoe to the river's edge and paddle his way to Winona for work.  Meanwhile, Winona State University Assistant Professor of Physics Nathan Moore is using the bridge closure to teach his beginning physics class the concept of relative velocity.  All Things Considered host Tom Crann asked Prof. Moore to help Jeremy Smith plan his commute. }
%
%\item 
%%\url{http://www.winonadailynews.com/articles/2008/03/23/opinion/otherviews/guest.txt}{\textbf{``Im-peck-able logic: The case for keeping chickens in Winona''}},  
%\textbf{``Im-peck-able logic: The case for keeping chickens in Winona''},
%Guest Opinion, Winona Daily News, 23 March, 2008.
%	
%
%\item 
%%\url{http://toulouse.physics.winona.edu/dokuwiki/doku.php?id=occasional_seminar:start}{
%\textbf{``Knot Populations in DNA''}
%%}
%, WSU Physics Department Occasional Seminar, January 2008
%%	
%%\textit{\textbf{Abstract:} Like shoelaces or fishing-line, DNA can be modeled as a long flexible strand, that often is tangled. Although nature has figured out a way to untangle DNA enough for it to be used biologically, little is known about the sort of tangles that form in vivo. Further, the details of how the meter of DNA in each cell is packed up into a cell are still ambiguous. In this talk I will describe the sort of knots that have been seen in some short DNA strands and describe some ideas for future student research projects at WSU.}
%
%\item 
%%\url{http://toulouse.physics.winona.edu/dokuwiki/doku.php?id=athenaeum_talk:start}{
%\textbf{``Using Wikis in Physics Courses,''}
%%}
%	WSU Library Athenaeum, 21 February, 2007 
%%
%%	\textit{\textbf{Abstract:} 
%%	A Wiki is a relatively new form of web document, 
%%	which allows visitors to modify page content, hopefully for the better, 
%%	with the noble ambition of extending the benefits of peer-review to the masses. 
%%	In the lecture I will describe my use of wikis in several Winona State physics 
%%	courses and share thoughts about the comparative strengths of this sort of course webpage.}
%	
%\item 
%%\url{http://toulouse.physics.winona.edu/dokuwiki/doku.php?id=hyvee:start}{
%\textbf{``The Science of Shopping.''} 
%%}
%	WSU's Consortium for Liberal Arts and Science Promotion,  25 October, 2006.
%%
%%	\textit{\textbf{Abstract:} 
%%	The problem of finding an efficient trip through the grocery 
%%	store is never so painfully acute as when I decide to pick up ``a few things'' at the store 
%%	and errantly chose Saturday morning, 10am to visit the Winona HyVee.  My talk will discuss 
%%	the problem of finding a fast route through an idealized grocery store in the context of 
%%	insights provided by statistical physics.  As time allows, I will describe how finding 
%%	one's way through the grocery store is quite similar to understanding how proteins fold.}
%
%	
%\item \textbf{``The presence of knots in DNA: How our bodies untangle fishing line.''}
%	Minnesota Science Teachers Association Regional Meeting, WSU, April 2006.
%%
%%	\textit{\textbf{Abstract:}
%%	Of interest to anglers seeking to fill their creels and children seeking to fasten their 
%%	shoes, knots have been compelling from time immemorial to those wishing to constrain a 
%%	string. In addition to the normal forces which bind molecules together, molecular chains 
%%	can also be knotted.  In the scientific community, knots were thought 100 years ago 
%%	to be the essence of matter.  More recently knots have been observed in and tied into DNA.  
%%	In the talk I will describe my own research work in the effects of knotting on polymer chains}
%
%\end{itemize}
%
%%------------------------------------------------------------------------------
%
%\section{\bf Student Research Projects and Internships}
%\begin{itemize}
%\item \textbf{Ryan Morgans}, coil gun testing and analysis, 2020-21.
%\item \textbf{Robert Zlock}, MIT coffee can radar 2019-20.
%\item \textbf{Adam Muschler}, photovoltaic-powered water pumping, 2018-19.
%\item \textbf{Noah Finn}, consturction and initial data from a ``Radio Jove'' phased dipole radio telescope, 2015-16.
%\item \textbf{Sam Robinson}, construction and initial data from a Stanford ``SuperSID'' space weather monitor, Spring 2016.
%\item \textbf{Keegan Christensen} Zigbee/Xbee antenna field strength and pattern analysis.  Presented at WSU Physics Occasional Seminar, January 2014.
%\item \textbf{Jodi Misar} Increases in pedagogical sophistication as the result of Science Club Facilitation, Fall 2012-Spring 2013.  Presented at the WSU Research Celebration, Spring 2013.
%\item \textbf{Megan Reiner} Arduino and Labview programming to implement an acoustical and kinesthetic baby monitor, Spring 2013, presented at MN-AAPT meeting, April 2013.
%\item \textbf{Megan Reiner, Tia Troy, and Tucker Besel} Labview and Arduino programming as an alternate lab curriculum in Physics 221, with Andrew J. Haugen.  Presented at MN-AAPT meeting, April 2013.
%\item \textbf{Anthony Martino}  Parallel and Graphics Card Programing, and Knot Invariant implementation (with Dr. Eric Errthum), Fall 2012.
%\item \textbf{Scott Stroh,} ``On the MPEX and the effects of using the Arduino Microcontroller in Introductory Physics Classes."  Spring 2011, presented at MN-AAPT meeting. 
%\item \textbf{Physics 222 class project,} VPython models of magnetic materials,
%Fall 2009.
%\item \textbf{Physics 222 class project,} VPython models of thermal transfer in a blacksmith's forge, 
%Fall 2008.
%\item \textbf{Mike Thornton}, ``Developing and Assessing Physics Programming Problems in VPython'', Spring-Fall 2008.
%\item \textbf{Greg Taubel}, ``Scientific Programming with the NVIDIA CUDA library'', Spring 2008-present.
%\item \textbf{Nick Szulcweski}, 
%``Numerical Weather Modelling with WRF-ARW'', Spring 2007 to Fall 2008.
%\item \textbf{Physics 221 class research project}, ``Applying Physics to problems in Home Arts''
%\item \textbf{Brian From}, 
%``Impact Craters and Pattern Recognition'', Fall 2007.
%\item \textbf{Physics 345 class research project}, Spring 2006.
%\item 
%
%\end{itemize}
%
%%------------------------------------------------------------------------------
%
%\section{\bf Service}
%\begin{itemize}
%\item LonCapa network particiant and problem author \url{https://loncapa.winona.edu}
%\item (Fall 2011-Fall 2016) Coordinated after-school ``Science Clubs" at Jefferson, Madison, Washington-Kosciusko,  and St. Stanislaus Elementary schools via WSU students in Phys180, EFRT099, OR100, and EDUC335.  Club activities are weekly practice with Cognitive Acceleration and Engineering is Elementary materials.
%%\item Coordinated an ``Occasional Physics Seminar", for the physics department (students and
%%faculty) and other interested parties.  
%%Generally attended by 10-20 students and faculty.
%\item WSU Academic Affairs and Curriculum Committee%, Fall 2006-present
%\item WSU Course Proposal and Program Committee (CPPS)%, Fall 2008-present
%\item South-East Minnesota Regional Science Fair, Organizing Committee, Spring 2007 -- 2015, Fair director in 2011.
%\item Search Committees for Dean of Library, Vice-President of IT/CIO, and Physics Faculty.
%\end{itemize}
%
%

%------------------------------------------------------------------------------
\section{\bf Grants and Workshops}

(2019) Winona State Digital Faculty Fellow: set up, develop problems for, and share awareness of the \url{https://www.lon-capa.org/} open-source homework system at Winona State. 

(2017) With Hannah Leverentz, $\approx\$25K$ to publicize ``Open Educational Resources," by organizing a series of \url{https://software-carpentry.org/} workshops at Minnesota State institutions.

(Summers 2012, 2013, \& 2015) 
Modeling Instruction Workshop for secondary science teachers at Winona State University.

%(Spring 2011, with John Deming, WSU Chemistry) WSU ``Teach21" Group award, $\approx ~\$15K$ to organized a middle and elementary teacher training workshop for (among others) staff at the STEM option at Jefferson Elementary in Winona, MN.  The meeting featured training on Cognitive Acceleration curricular materials, and discussion on how to convert standard science lessons to (5E) inquiry format.
%
%(Spring 2011, with Bill Braun, Winona Senior High) WSU ``Bush Education Group" award, $\$7.5K$ to organized a Modeling Instruction Workshop for secondary science teachers at Winona State University, August 2011.
%
%(Fall 2009, with John Deming, WSU Chemistry)  MNSCU Discipline Workshop, $\approx \$3K$ to host an on-campus workshop (held in May, 2010) to disseminate the inquiry-based approach to communicating science content that has proved effective in WSU's first-year science cohort. In parallel, we offered a workshop for secondary science teachers. 
%
%(Fall 2008, with John Deming, WSU Chemistry) MNSCU CTL Grant, $\approx \$29K$ (with $\$5K$ matching from WSU) to fund a section of Chem 108, co-taught with John Deming of WSU Chemistry.  The course is formulated to increase retention rates among under-prepared freshman science majors through the development of thinking skills.
%
%(Summer-Fall 2008) MNSCU Learning Games and Simulations award.  $\approx \$20K$ to fund the creation of a suite of physics problems to be solved with VPython, an open-source programming language that produces interactive physics simulations.  Funds used for faculty salary and to pay a summer research salary to a WSU Physics Education major also working on the project.
%
%(Spring 2008) WSU IFO FDC Award, $\$500$ for the creation of a bread oven on campus (in the context of Physics 345, WSU's thermodynamics course).
%
%(Fall 2006) WSU IFO PIF Travel Award.  Visit to a training conference at NCAR (Colorado Boulder) with a WSU student to attend a WRF-ARW (numerical weather model) training conference.




%------------------------------------------------------------------------------



%%---------------------------------------------------------------------------%%

\section{\bf Professional Associations}
\begin{itemize}
%\item American Physical Society (2000-present)
\item American Association of Physics Teachers 
\item American Modeling Teachers' Association (AMTA, life member).
\item IEEE Senior member, (2016--present).
\item Certified Software Carpentry Instructor, Dec 2016.
\end{itemize}

\textit{This document was generated on \today}
\end{resume}

\end{document}

%%===========================================================================%%
